\chapter{Introdução}

Tendo em vista o desafio de prover Inclusão Digital, principalmente para
populações que não cresceram tendo acesso aos meios digitais, seja por falta de
recursos e oportunidade, seja por questões temporais, propomos o
desenvolvimento de uma plataforma educacional voltada para este público. A
ideia é prover cursos e guias de simples acesso para guiar o uso dos meios
digitais desde os primeiros fundamentos, promovendo instrução de uso,
conscientização e uma futura independência digital.

Sendo assim, tomamos como inspiração uma das maiores plataformas educacionais
de livre acesso hoje disponíveis, a \textit{Khan Academy}, porém mudando seu
propósito para além do mundo acadêmico e tentando enfatizar a acessibilidade
àqueles que não estão habituados aos meios digitais.

\chapter{Modelo Entidade-Relacionamento}

\section{Levantamento de Requisitos}

Um usuário do sistema poderá utilizar a plataforma de forma anônima (não
autenticada) ou de maneira identificável. Para que isso seja possível, um
usuário pode se cadastrar de maneira simples \textbf{(usuário autenticado)},
inserindo apenas um \textbf{nome de usuário} (3 a 16 caracteres), uma
\textbf{senha} (que será armazenada como uma hash, estimando 144 caracteres
necessários para sua persistência) e, opcionalmente, um \textbf{e-mail}
(ocupando no máximo 254 caracteres). Além disso, para moderar a plataforma e
atualizá-la com mais conteúdos, teremos também \textbf{usuários
  administrativos}, esses são semelhante aos usuários convencionais, sendo
diferenciados por um booleano \textbf{admin}, além disso todo administrador
deve obrigatoriamente conter um e-mail cadastrado que será utilizado para
autenticação em dois fatores.

Um \textbf{administrador} pode banir um usuário da plataforma, quando isso
ocorre temos a entidade de \textbf{banimento}, nela são indicados: o
\textbf{responsável} (referência ao administrador) responsável pelo banimento,
o \textbf{usuário alvo} (referência ao usuário banido), o \textbf{horário}
(data e hora de criação), a \textbf{validade} (data e hora em que o banimento
expira) e, opcionalmente, a \textbf{causa} (texto) do banimento. Além disso,
temos algumas restrições de integridade, sendo elas: o \textit{responsável} tem
que ser um \textit{administrador} e o \textit{usuário alvo} não pode ser um
\textit{administrador}.

A ideia é que esse seja um projeto dirigido e voltado para a comunidade, por
conta disso é bastante importante que usuários consigam prover feedbacks,
sugerindo melhorias, apontando problemas encontrados e solicitando novos
tópicos. Esses feedbacks, então, terão uma referência de qual é o autor
(referência ao usuário que criou o feedback), a mensagem (texto de tamanho
dinâmico, limitado arbitrariamente à 4096 caracteres). Além disso, a fim de
garantir a validação dos feedbacks, temos uma entidade de visualização de
feedback, ela associa unicamente um leitor (referência ao usuário
administrador) e um feedback (referência ao feedback), incluindo as
informações: data de primeira abertura (data e hora da primeira abertura do
feedback por parte daquele administrador), data da última abertura (data e hora
da última abertura por parte daquele administrador) e a data de resolução (data
e hora indicando o momento em que aquele feedback foi marcado como resolvido
por aquele administrador, nulo quando não marcado como resolvido).

A fim de organizar o conteúdo, seguiremos uma divisão hierárquica bastante
comum, seguindo em: categorias, tópicos, unidades e aulas, do maior para o
menor, respectivamente. Primeiramente, teremos a categoria, sendo o nível mais
alto da hierarquia, ela é bastante simples, tendo apenas um nome (texto), uma
descrição (texto) e um ícone (url). Em sequência, temos os cursos, que contém
um nome (texto), uma descrição (texto) e a categoria (referência à categoria).
Seguindo adiante, teremos as unidades, nosso próximo nível na hierarquia. As
unidades seguem uma estrutura similar, contendo um nome (texto), uma descrição
(texto), sua ordem (inteiro), que é usada para controlar a ordem em que as
unidades são apresentadas, e o curso (referência à curso) a qual pertence.
Descendo mais um nível, temos os tópicos, a estrutura dos tópicos é também
semelhante aos anteriores, tendo um nome (texto), uma descrição (texto), e, por
fim, a unidade (referência a unidade) a qual pertence. Por último temos as
aulas, elas são o nível mais baixo na hierarquia organizacional, abrigando os
conteúdos em si. As aulas seguem praticamente a mesma estrutura das unidades,
ou seja, possuem um nome (texto), uma descrição (texto), uma ordem (inteiro) e,
por fim, o tópico (referência ao tópico) a qual pertence.

Um dos principais elementos da plataforma, que seria a linha final desta
hierarquia, são os conteúdos, todos os conteúdos terão alguns campos em comum:
um título (texto), um subtítulo (texto) e uma duração estimada (inteiro
indicando o número de minutos indicados). Além disso, inicialmente, teremos
três tipos de conteúdo: artigos, que seriam conteúdos primariamente textuais,
contendo, além dos comuns entre os conteúdos, o corpo do artigo (que será um
texto formatado em MD). Teremos também, os vídeos, que contém o link do vídeo
(url) e uma descrição (texto) do vídeo. Por fim, teremos os exercícios, eles
são compostos de um enunciado (texto formato em MD), um conjunto de
alternativas (quantidade variável, podendo ter 0), limite de seleções de
alternativas (inteiro indicando quantas alternativas um usuário pode
selecionar). Cada alternativa consiste simplesmente um corpo (texto formatado
em MD), opcionalmente, uma explicação (texto formatado em MD) e se é uma
alternativa correta (booleano). Usar uma combinação de exercício e texto para
identificar unicamente alternativas.

Para usuários autenticados, será mantido um histórico do progresso dele na
plataforma, isto é: para cada conteúdo, será registrado um evento de progresso,
indicando o horário onde foi feito o progresso, qual o conteúdo (referência ao
conteúdo) que foi visitado e, caso o conteúdo seja um exercício, as
alternativas selecionadas (referências às alternativas). Importante notar que
as alternativas selecionadas precisam fazer parte do conteúdo do tipo exercício
referente ao evento. Os eventos de progresso podem ser visualizados pelo
usuário e podem existir mais de um evento para o mesmo conteúdo. O usuário pode
deletar eventos antigos a fim de reiniciar o seu progresso de acordo com a
granularidade desejada.

Além disso, usuários autenticados, podem deixar comentários em conteúdos, cada
comentário é composto por um corpo (texto em MD), o autor (referência ao
usuário que o criou), o conteúdo (referência ao conteúdo) em que o comentário
foi feito, o horário (data e hora de criação) em que o horário foi feito, se
ele é visível (booleano indicando se ele é exibido) para casos de deleção e,
por fim, um comentário pode ser uma resposta, nesses casos temos o campo de
comentário pai (referência opcional a um comentário) que indica qual comentário
aquele responde. A presença desse mecanismo de respostas cria uma restrição de
integridade onde precisamos garantir que, caso o comentário seja uma resposta,
o conteúdo associado ao comentário pai seja o mesmo que o conteúdo referenciado
presente em si. Adicionar motivo.

A fim de garantir a moderação da plataforma, usuários autenticados podem
reportar comentários, para isso temos a entidade de report. Cada report
consiste de um autor (referência ao usuário que fez o report), o comentário
reportado (referência ao comentário sendo reportado), o horário (data e hora de
criação) e, por fim, o indicador de verificado (booleano), que indica que
aquele report já foi avaliado por um administrador.
