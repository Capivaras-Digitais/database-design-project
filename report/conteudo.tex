\chapter{Introdução}

Tendo em vista o desafio de prover Inclusão Digital, principalmente para
populações que não cresceram tendo acesso aos meios digitais, seja por falta de
recursos e oportunidade, seja por questões temporais, propomos o
desenvolvimento de uma plataforma educacional voltada para este público. A
ideia é prover cursos e guias de simples acesso para guiar o uso dos meios
digitais desde os primeiros fundamentos, promovendo instrução de uso,
conscientização e uma futura independência digital.

Sendo assim, tomamos como inspiração uma das maiores plataformas educacionais
de livre acesso hoje disponíveis, a \textit{Khan Academy}, porém mudando seu
propósito para além do mundo acadêmico e tentando enfatizar a acessibilidade
àqueles que não estão habituados aos meios digitais.

\chapter{Modelo Entidade-Relacionamento}

%%%%%%%%%%%%%%%%%%%%%%%%%%%%%%%%%%%%%%%%%%%%%%%%%%%%%%%%%%%%%%%%
% Requisitos

\section{Levantamento de Requisitos}

%%%%%%%%%%%%%%%%
% Autenticação %
%%%%%%%%%%%%%%%%

Um usuário do sistema poderá utilizar a plataforma de forma anônima (não
autenticada) ou de maneira identificável. Para que isso seja possível, um
usuário pode se cadastrar de maneira simples \textbf{(usuário autenticado)},
inserindo apenas um \textbf{nome de usuário} (3 a 16 caracteres), uma
\textbf{senha} (que será armazenada como uma hash, estimando 144 caracteres
necessários para sua persistência) e, opcionalmente, um \textbf{e-mail}
(ocupando no máximo 254 caracteres). Além disso, para moderar a plataforma e
atualizá-la com mais conteúdos, teremos também \textbf{usuários
  administrativos}, esses são semelhante aos usuários convencionais, sendo
diferenciados por um booleano \textbf{admin}, além disso todo administrador
deve obrigatoriamente conter um e-mail cadastrado que será utilizado para
autenticação em dois fatores.

%%%%%%%%%%%%%%
% Banimentos %
%%%%%%%%%%%%%%

Um \textbf{administrador} pode banir um usuário da plataforma, quando isso
ocorre temos a entidade de \textbf{banimento}, nela são indicados: o
\textbf{responsável} (referência ao administrador) responsável pelo banimento,
o \textbf{usuário alvo} (referência ao usuário banido), o \textbf{horário}
(data e hora de criação), a \textbf{validade} (data e hora em que o banimento
expira) e, opcionalmente, a \textbf{causa} (texto) do banimento. Além disso,
temos algumas restrições de integridade, sendo elas: o \textit{responsável} tem
que ser um \textit{administrador} e o \textit{usuário alvo} não pode ser um
\textit{administrador}.

%%%%%%%%%%%%%
% Feedbacks %
%%%%%%%%%%%%%

A ideia é que esse seja um projeto dirigido e voltado para a comunidade, por
conta disso é bastante importante que usuários consigam prover
\textbf{feedbacks}, sugerindo melhorias, apontando problemas encontrados e
solicitando novos tópicos. Esses feedbacks, então, terão uma referência de qual
é o \textbf{autor} (referência ao usuário que criou o feedback), a
\textbf{mensagem} (texto de tamanho dinâmico, limitado arbitrariamente à 4096
caracteres) e, opcionalmente, a \textbf{data de resolução} (data e hora quando
o feedback foi marcado como resolvido). Além disso, a fim de garantir a
validação dos feedbacks, temos uma entidade de \textbf{visualização de
  feedback}, ela associa um \textbf{leitor} (referência ao usuário administrador)
e um \textbf{feedback} (referência ao feedback) junto à informação da
\textbf{data de visualização} (data e hora quando a visualização foi feita).

%%%%%%%%%%%%%%%
% Organização %
%%%%%%%%%%%%%%%

A fim de organizar o conteúdo, seguiremos uma divisão hierárquica bastante
comum, seguindo em: categorias, tópicos, unidades e aulas, do maior para o
menor, respectivamente. Primeiramente, teremos a \textbf{categoria}, sendo o
nível mais alto da hierarquia, ela é bastante simples, tendo apenas um
\textbf{nome} (texto), uma \textbf{descrição} (texto) e um \textbf{ícone}
(url). Em sequência, temos os \textbf{cursos}, que contém um \textbf{nome}
(texto), uma \textbf{descrição} (texto) e a \textbf{categoria} (referência à
categoria). Seguindo adiante, teremos as \textbf{unidades}, nosso próximo nível
na hierarquia. As unidades seguem uma estrutura similar, contendo um
\textbf{nome} (texto), uma \textbf{descrição} (texto), sua \textbf{ordem}
(inteiro), que é usada para controlar a ordem em que as unidades são
apresentadas, e o \textbf{curso} (referência à curso) a qual pertence. Descendo
mais um nível, temos os \textbf{tópicos}, a estrutura dos tópicos é também
semelhante aos anteriores, tendo um \textbf{nome} (texto), uma
\textbf{descrição} (texto), e, por fim, a \textbf{unidade} (referência a
unidade) a qual pertence. Por último temos as \textbf{aulas}, elas são o nível
mais baixo na hierarquia organizacional, abrigando os conteúdos em si. As aulas
seguem praticamente a mesma estrutura das unidades, ou seja, possuem um
\textbf{nome} (texto), uma \textbf{descrição} (texto), uma \textbf{ordem}
(inteiro) e, por fim, o \textbf{tópico} (referência ao tópico) a qual pertence.

%%%%%%%%%%%%%
% Conteúdos %
%%%%%%%%%%%%%

Um dos principais elementos da plataforma, que seria a linha final desta
hierarquia, são os \textbf{conteúdos}, todos os conteúdos terão alguns campos
em comum: um \textbf{título} (texto), um \textbf{subtítulo} (texto) e uma
\textbf{duração estimada} (inteiro indicando o número de minutos indicados).
Além disso, inicialmente, teremos três tipos de conteúdo: \textbf{artigos}, que
seriam conteúdos primariamente textuais, contendo, além dos comuns entre os
conteúdos, o \textbf{corpo} do artigo (que será um texto formatado em MD).
Teremos também, os \textbf{vídeos}, que contém a \textbf{url} do vídeo (texto)
e uma \textbf{descrição} (texto) do vídeo. Por fim, teremos os
\textbf{exercícios}, eles são compostos de um \textbf{enunciado} (texto formato
em MD), um conjunto de \textbf{alternativas} (quantidade variável, podendo ter
0), \textbf{limite de seleções de alternativas} (inteiro indicando quantas
alternativas um usuário pode selecionar). Cada \textbf{alternativa} consiste
simplesmente um \textbf{corpo} (texto formatado em MD), opcionalmente, uma
\textbf{explicação} (texto formatado em MD) e se é uma alternativa
\textbf{correta} (booleano).

%%%%%%%%%%%%%
% Histórico %
%%%%%%%%%%%%%

Para \textit{usuários autenticados}, será mantido um histórico do progresso
dele na plataforma, isto é: para cada \textit{conteúdo}, será registrado um
\textbf{evento de progresso} (histórico no MER), indicando o \textbf{horário}
(data e hora) onde foi feito o progresso, qual o \textbf{conteúdo} (referência
ao conteúdo) que foi visitado e, caso o conteúdo seja um \textit{exercício}, as
\textbf{alternativas selecionadas} (referências às alternativas). Importante
notar que as alternativas selecionadas precisam fazer parte do conteúdo do tipo
exercício referente ao evento. Os eventos de progresso podem ser visualizados
pelo usuário e podem existir mais de um evento para o mesmo conteúdo. O usuário
pode deletar eventos antigos a fim de reiniciar o seu progresso de acordo com a
granularidade desejada.

%%%%%%%%%%%%%%%
% Comentários %
%%%%%%%%%%%%%%%

Além disso, \textit{usuários autenticados}, podem deixar comentários em
conteúdos, cada \textbf{comentário} é composto por um \textbf{corpo} (texto em
MD), o \textbf{autor} (referência ao usuário que o criou), o \textbf{conteúdo}
(referência ao conteúdo) em que o comentário foi feito, o \textbf{horário}
(data e hora de criação) em que o horário foi feito, se ele é \textbf{visível}
(booleano indicando se ele é exibido) para casos de deleção e, por fim, um
comentário pode ser uma resposta, nesses casos temos o campo de
\textbf{comentário pai} (referência opcional a um comentário) que indica qual
comentário aquele responde. A presença desse mecanismo de respostas cria uma
restrição de integridade onde precisamos garantir que, caso o comentário seja
uma resposta, o conteúdo associado ao comentário pai seja o mesmo que o
conteúdo referenciado presente em si. Além disso, a fim de evitar ciclos de
relacionamento, precisamos garantir que, ao ir seguindo a cadeia de comentários
pai, não haja a existência de um ciclo no momento da criação.

%%%%%%%%%%%
% Reports %
%%%%%%%%%%%

A fim de garantir a moderação da plataforma, \textit{usuários autenticados}
podem reportar comentários, para isso temos a entidade de \textbf{report}. Cada
report consiste de um \textbf{autor} (referência ao usuário que fez o report),
o \textbf{comentário reportado} (referência ao comentário sendo reportado), o
\textbf{horário} (data e hora de criação) e, por fim, o indicador de
\textbf{verificado} (booleano), que indica que aquele report já foi avaliado
por um administrador.

% Talvez: adicionar derivação de progresso nas categorias, tópicos...

%%%%%%%%%%%%%%%%%%%%%%%%%%%%%%%%%%%%%%%%%%%%%%%%%%%%%%%%%%%%%%%%
% Funcionalidades

\section{Funcionalidades}

Analisando por cima dos atores do sistema, conseguimos sintetizar as
funcionalidades da plataforma da seguinte forma:

\begin{itemize}
  \item \textbf{Usuário Não Cadastrado}
        \begin{itemize}
          \item Cadastrar-se
          \item Navegar pelas categorias, cursos, tópicos, unidades e conteúdos
          \item Consumir os conteúdos
          \item Realizar exercícios
          \item Visualizar comentários de um conteúdo
        \end{itemize}
  \item \textbf{Usuário Cadastrado}
        \begin{itemize}
          \item \textit{(Todas de um Usuário Não Cadastrado)}
          \item Autenticar-se
          \item Alterar senha
          \item Atualizar e-mail
          \item Registrar (automaticamente) o progresso na plataforma
          \item Visualizar o progresso atual
          \item Reiniciar o progresso (de maneira granular)
          \item Criar comentários em um conteúdo (caso não esteja banido)
          \item Responder comentários existentes (caso não esteja banido)
          \item Reportar um comentário (caso não esteja banido)
          \item Deletar um próprio comentário
          \item Enviar feedbacks (caso não esteja banido)
        \end{itemize}
  \item \textbf{Usuário Administrador}
        \begin{itemize}
          \item \textit{(Todas de um Usuário Cadastrado)}
          \item Inserir, remover e atualizar: categorias, cursos, tópicos, unidades, artigos,
                aulas, exercícios e alternativas
          \item Deletar e atualizar comentários (de outros usuários)
          \item Banir e perdoar usuários cadastrados (que não sejam administradores)
          \item Visualizar os reports e comentários associados
          \item Marcar reports de um comentário como resolvido (todos até o momento atual)
          \item Visualizar feedbacks (histórico de visualização persistido)
          \item Marcar (ou desmarcar) feedback como resolvido (mantendo a data de resolução)
        \end{itemize}
\end{itemize}

%%%%%%%%%%%%%%%%%%%%%%%%%%%%%%%%%%%%%%%%%%%%%%%%%%%%%%%%%%%%%%%%
% Ciclos e restrições de integridade

\section{Ciclos e Restrições de Integridade}
